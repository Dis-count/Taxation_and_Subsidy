\section*{Abstract}



Keywords: Cooperative game, scheduling, pricing for coopration.


In this paper, instead of using the penalization and subsidization instrument to promote the cooperation, we are interested in finding a way to stabilize the grand coalition by introducing the new instrument, pricing. Inspired by the setup cost of machines in scheduling problems, the setup cost can be considered as a kind of price intuitively. Note that the original global schedule could be changed as the price changes. Meanwhile, we could use the price increment as the subsidy to stabilize the grand coalition in the “forced” collaborative schedule.
To promote the collaboration among players in the grand coalition without any externally-funded subsidy, it is of great importance to provide central authority with a method to set the price.

Our main contributions are summarized as follows.

First, to our best knowledge, this paper is the first attempt to introducing a brand new concept pricing to the scheduling cooperative game. Most literature on cooperative game, especially on scheduling or sequencing cooperative game, focuses on the importance
and
And recently studied the joint effects of subsidization and penalization on the coalition to stabilize the grand coalition.

Our study provides another perspective to help the authority adopt an mechanism setting the price to promote cooperation without any externally-funded subsidy.

Second, we establish the model to analyze the effects of pricing on the cooperative game, and we characterize the “waiting cost” to explain the inequity between investors at different stages in crowdfunding projects.
The main results show that because of the waiting cost, investors who arrive early are less willing to pledge money. It also shows that the entrepreneur should motivate early investors to enhance the success rate of the project. In addition, the extra return given to early investors as an incentive should increase with the waiting cost.

Third, as a generalization, we consider the difference in the number of investors who group as cohorts, arriving at different points in time. We find that investors in different-sized cohorts are not equally sensitive with changes in profit allocation, and the entrepreneur should motivate investors in smaller cohorts to enhance the success rate of his crowdfunding project. This property, together with the effect of the waiting cost, decides the profit allocation strategy of the entrepreneur. In addition, we also provide managerial guidance on how the entrepreneur should adjust the optimal profit allocation mechanism when other factors in the market change.
三方面贡献,首先最重要的 是提出 pricing 的概念, 这个概念不同于已有的 惩罚和补贴;
提出概念之后建立了相应的模型,并且提出了相应的解法,归到一个可以求解的问题;
generalization, 对于有权重的部分也可用类似的方法,得到了能这样求解的一般性质;
pricing 对于政府定价有一定的启示.

以及第七的拓展部份 提到 pricing 会有固定的两/三个价格提供稳定  不存在 机器数量不变/scheduling的情况下, 可以稳定.

The rest of this paper is structured as follows. The following section reviews relevant literature. We describe the motivating problem in Section 3. In Section 4, we establish the model and analyze its properties. Section 5 discusses the generalization results of Section 4 and gives the method to determine the price. The conclusions are shown in Section 6.
