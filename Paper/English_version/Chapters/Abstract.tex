\section*{Abstract}



Keywords: Cooperative game, scheduling, pricing for coopration.


In this paper, instead of sacrificing the entrepreneur’s profits, we are interested in motivating early investors by reallocating final profits earned from the proposal. Intuitively, we assign more profits to early investors so that their waiting costs are balanced out and the resulting pledging probabilities are raised. Note that more profits allocated to (higher pledging probabilities of) early investors means less profits remain for (lower pledging probabilities of) the late ones. To enhance the overall success rate of a crowdfunding project, it is of utmost importance to provide the entrepreneur with appropriate profit allocation mechanisms. Our main contributions are summarized as follows.
First, to the best of our knowledge, this paper is the first attempt to analytically study the profit
allocation mechanism to enhance the success rates of investment-based crowdfunding projects. Most
literature on crowdfunding, especially investment-based crowdfunding, is empirical and existing efforts
on motivating investors focus on offering additional benefits and price discounts. Our study helps the entrepreneur design an optimal profit allocation mechanism to maximize the success rate without sacrificing the profits of the entrepreneur.
Second, we develop static models to analyze the pledging behavior of investors, and we characterize
the “waiting cost” to explain the inequity between investors at different stages in crowdfunding projects. The main results show that because of the waiting cost, investors who arrive early are less willing to pledge money. It also shows that the entrepreneur should motivate early investors to enhance the success rate of the project. In addition, the extra return given to early investors as an incentive should increase with the waiting cost.
Third, as a generalization, we consider the difference in the number of investors who group as cohorts, arriving at different points in time. We find that investors in different-sized cohorts are not equally sensitive with changes in profit allocation, and the entrepreneur should motivate investors in smaller cohorts to enhance the success rate of his crowdfunding project. This property, together with the effect of the waiting cost, decides the profit allocation strategy of the entrepreneur. In addition, we also provide managerial guidance on how the entrepreneur should adjust the optimal profit allocation mechanism when other factors in the market change.
三方面贡献,首先最重要的 是提出 pricing 的概念, 这个概念不同于已有的 惩罚和补贴; 提出概念之后建立了相应的模型,并且提出了相应的解法,归到一个可以求解的问题; generalization, 对于有权重的部分也可用类似的方法,得到了能这样求解的一般性质;
pricing 对于政府定价有一定的启示.

以及第七的拓展部份 提到 pricing 会有固定的两/三个价格提供稳定

The rest of this paper is structured as follows. The following section reviews relevant literature. We describe the basic problem in Section 3. In Section 4, we analyze the profit allocation mechanism using a
primary model where there are only two potential investors. Section 5 generalizes the results of Section 4 by studying a two-cohort model where there are two cohorts of investors. The conclusions are shown in Section 6.
