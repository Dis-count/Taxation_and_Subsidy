\section*{Introduction}

At first, we will introduce some preliminary knowledge about cooperative game theory as follows.
A pair $(V,c)$ is usually used to represent a cooperative game with transferable utilities, among which $V=\{1,2,\dots,v\}$ denotes a set of $v$ players and $c:2^{V}\to \mathbb{R}$ indicates the characteristic function of the game. A coalition $s$ is defined as a non-empty subset of players; and $V$ is refered to as the grand coalition containing all the players, $S=2^{V} \setminus\{\emptyset\}$ denotes the set of all feasible coalitions. The characteristic function of the game, $c(s)$, represents the minimum coalitional cost the players in $s$ have to pay in order to cooperate together.
A cost allocation vector $\alpha=[\alpha_{1},\alpha_{2},\dots,\alpha_{v}] \in \mathbb{R}^{v}$ is required by the game to maintain cooperation in the grand coalition, and $\alpha_{k}$ is the cost assigned to players $k \in V$ to assure no individual or group of players has the incentive to deviate. For the convenience of abbreviation, we use $\alpha(s)=\sum_{k\in{s}}\alpha_{k}$ to denote the total cost assigned to the players in coalition $s$.
One of the most important concepts in cooperative game theory is core, which is a cost allocation satisfying two kinds of contraints, one is the budget balance contraint {$\alpha(V)=c(V)$} and the other is the coalitional stability constraints {$\alpha(s) \leq c(s)$}. In other words, core can be expressed as

\[
Core(V,c)= \left\{\alpha:\alpha(V)=c(V), \alpha(s)\leq c(s)\ \text{for all}\ s \in S \setminus\{V\}, \alpha \in \mathbb{R}^{v} \right\}.
\]

When the cost allocation exists, core is called non-empty. And if and only if the core is non-empty, the grand coalition of the associating cooperative game $(V,c)$ will be stable or balanced.

However, cooperative games can be unbalanced in many cases owing to the joint restrictions of the above-mentioned two kinds of constraints. To stabilize the grand coalition in unbalanced cooperative games, researchers have already developed several effective instruments, such as subsidization, penalization and simultaneously subsidization and penalization. The
similarity of these instruments is that there exists an outside party who will take measures to stabilize the grand coalition. But the penalization will always arouse the discontent of players in coalitions, it is promising to find a new instrument to replace the penalization, we call it pricing.
The significant idea of this instrument is that we can erase the role of the third party by collecting pricing as subsidy of the corresponding coalitions.
That being said, the players of the games can stabilize themselves without the third party.

To make the project concrete, we will apply this instrument on the machine scheduling games to show our ideas.
