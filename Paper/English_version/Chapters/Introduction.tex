% !TEX root = sum1.tex
\section{Introduction}

In this paper, instead of using the penalization and subsidization instrument to promote the cooperation, we are interested in finding a way to stabilize the grand coalition by introducing the new instrument, pricing. Inspired by the setup cost of machines in scheduling problems, the setup cost can be considered as a kind of price intuitively. Note that the original global schedule could be changed as the price changes. Meanwhile, we could use the price increment as the subsidy to stabilize the grand coalition in the “forced” collaborative schedule.
To promote the collaboration among players in the grand coalition without any externally-funded subsidy, it is of great importance to provide the central authority with a method to set the price.

Our main contributions are summarized as follows.

First, to our best knowledge, this paper is the first attempt to introducing a brand new concept pricing to the scheduling cooperative game. Most literature on scheduling or sequencing cooperative game, highlight the importance of cooperation in sequencing and scheduling, there is not much work on stabilizing a grand coalition when the core is empty. Recently, some scholars studied the joint effects of subsidization and penalization on the empty-core cooperative game to stabilize the grand coalition. Especially, Our study provides another perspective to help the authority adopt a mechanism setting the price to promote cooperation without any externally-funded subsidy.

Second, we establish the model to analyze the effects of pricing on the cooperative game, and we characterize the price to explain the taxation on the players in different coalitions. However, rather than arousing dissatisfaction among players by taxation, the price increment will be used as a subsidy to promote cooperation. Then, We develop the theorems to guide us to design effective algorithms to solve this problem.

Third, as a generalization, we apply this method on the weighted one and show the similar conclusions. With this new instrument, we illustrate how to select the specific price by the central authority to stabilize the coalition. In addition, we also provide managerial guidance for the government on how to set a reasonable price to promote cooperation and no need to pay extra costs.


The rest of this paper is structured as follows. The following section reviews relevant literature. We describe the motivating problem in Section 3. In Section 4, we establish the model and analyze its properties. Section 5 demonstrates the algorithms for sovling IVPU game. Section 6 gives the extension of the game. The conclusions are shown in Section 7.
