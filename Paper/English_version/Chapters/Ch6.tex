\section*{Algorithm for solving IVPU game}

At first, we need to construct the PSPF function, which is piecewise linear and convex at every subinterval. If we can obtain the specific subsidy value $\omega(P)$ given any $P$, then we can calculate the breakpoints during the interval by IPC algorithm.

At first, we need to divide the interval into $n$ subintervals according to the  number of using machine of the grand coalition, at each subinterval we use the IPC algorithm to get the breakpoints. After applying the algorithm to all subintervals, we can obtain the PSPF function.


\begin{algorithm}[h]\label{algoIPC}
\caption{The Intersection Points Computation(IPC) Algorithm to Construct the PSPF Function.}
\begin{algorithmic}[1]

\begin{description}
  \justifying
  \item[Step 1.] Initially, set $I^*=\{P_L,P_H\}$ and $\mathbb{I}= \{[P_L,P_H]\}$.
  \item[Step 2.] If $\mathbb{I}$ is not empty, update $I^*$ and $\mathbb{I}$ by the following steps:
  \item[Step 3.] Sort values in $I^*$ by $P_0<P_1<\cdots<P_q$, where $P_0 = P_L,P_q = P_H$ and $q = |I^*|-1$.
  \item[Step 4.]
  Select any interval from $\mathbb{I}$, denoted by $[P_{k-1},P_{k}]$ with $1\leq k \leq q$
  \item[Step 5.]
  Construct two linear function $ R_{k-1}(P)$ and $ L_{k}(P)$ so that $ R_{k-1}(P)$ passes $(P_{k-1},\omega(P_{k-1}))$ with \\
  \vspace{10pt}
  a slope equal to a right derivative $K_{r}^{P_{k-1}}$ of $\omega(P)$ at $P_{k-1}$, and that $L_{k}(z)$ passes $(P_{k},\omega(P_{k}))$ with a \\
  \vspace{10pt}
  slope equal to a left derivative $K_{l}^{P_{k}}$
  of $\omega(P)$ at $P_k$.
  \item[Step 6.] If $R_{k-1}(P)$ passes $(P_{k},\omega(P_{k}))$ or $L_{k}(P)$ passes $(P_{k-1},\omega(P_{k-1}))$, then update $\mathbb{I}$ by removing \\

  $[P_{k-1},P_{k}]$. Otherwise, $R_{k-1}(P)$ and $L_{k}(P)$ must have a unique intersection point at $P=P'$ for  \\
  \vspace{10pt}
  some $P' \in (P_{k-1},P_{k})$.
  Update $I^*$ by adding $P^'$, and update $\mathbb{I}$ by removing $[P_{k-1},P_{k}]$, adding \\
  \vspace{10pt}
  $[P_l,P']$ and $[P',P_r]$.
  \item[Step 7.] Go to step 2.
  \item[Step 8.] Return a piecewise linear function by connecting points $(P,\omega(P))$ for all $P \in I^*$.

\end{description}

\end{algorithmic}
\end{algorithm}

Then we need to calculate the $\omega(P)$ when given an arbitrary $P$. We can follow the basic Cutting Plane(CP) algorithm, the specific process are represented in algorithm 2.

\begin{algorithm}[h]\label{algoCP}
\caption{The Cutting Plane(CP) Algorithm to compute $\omega(P)$ for a given $P$.}
\begin{algorithmic}[1]

\begin{description}
  \justifying
  \item[Step 1.] Let $\mathbb{S}'\subseteq \mathbb{S}\setminus \{N\}$ indicates a restricted coalition set, which includes some initial coalitions,
  \vspace{10pt}
  e.g.,$ \{1\},\{2\},\ldots,\{v\}$.
  \item[Step 2.] Find an optimal solution $\bar{\alpha}(\ \cdot \ ,P)$ to LP $\tau(P)$:
  \begin{equation*}
  \max_{\alpha\in \mathbb{R}^n} \big\{ \alpha(N,P): \alpha(s,P) \leq c(s)+P, \mbox{ for all } s \in \mathbb{S}'\big\}.
  \end{equation*}
  \vspace{-11pt}
  \item[Step 3.]
  Find an optimal solution $s^*$ to the separation problem:
  \begin{equation*}
  \delta = \min \big\{ c(s)+ P -\bar{\alpha}(s,z): \forall s \in \mathbb{S} \setminus \{N\}\big\}.
  \end{equation*}
  \item[Step 4.]
  If $\delta<0$, then add $s^*$ to $\mathbb{S}'$, and go to step 2; otherwise, return $\omega(P)=c(N)-\bar{\alpha}(N,P)$ and the pair of derivatives $(K_{l}^{\bar{\beta}},K_{r}^{\bar{\beta}})$.
\end{description}

\end{algorithmic}
\end{algorithm}

As we all know, the cost arised from the partial players in the grand coalition, that is $c(s)$, can be calculated handily by the SPT rule (The corresponding conclusion see Lemma \ref{lem1}). Meanwhile, inspired by the paper (Please refer to Liu et.al.2018), we have the following approach to solve this game.

There is an effective domain $[0, P^*]$, where the grand coalition may need a subsidy from the external to maintain the balance.
(The corresponding conclusion see Theorem \ref{thm1} and \ref{thm2} where $P^* = P_L(V,1)$).

For the effective domain, we just need to divide this interval into several parts and the breakpoints are denoted as $P_L(V,i), i = 1,2,\ldots,n$.  At first, we don't need to calculate the initial part because at this part the corresponding subsidy is 0 always. (The corresponding conclusion see Theorem \ref{thm3}) Then we just need to focus on the latter part which shows some interesting properties we present above.

As we've already known that $P_L(V,i), i = 1,2,\ldots,n$ can be obtained by Lemma \ref{lem1} and Theorem \ref{thm2}, we just need to follow the CP approach (Algorithm 3 in Liu et.al.2018) to calculate the weak derivatives at each sub-interval $[P_L(m,V),P_H(m,V)]$ where the corresponding derivative is $m_V-\sum_{s\in S\setminus\{V\}} \rho_s$.

Then use IPC Algorithm which will return all the breakpoints during the $[0, P^*]$ to obtain the subsidy $\omega(P)$.

Notice that we can calculate the characteristic function $c(s)$ easily according to Theorem \ref{thm5}, we can formulate Theorem \ref{thm6}.

\begin{lem}\label{lem5}
  For the IVPU game, the corresponding separation problem can be sovled in $O(v^2)$ time, which could be shown in the following DP algorithm.
\end{lem}

\begin{algorithm}[h]\label{algoDP}
\caption{The Dynamic Programming(DP) Algorithm to Solve the seperation problem.}
\begin{algorithmic}[1]

% the coalition $s$ is a subset of $\{1,2,\ldots,k\}$.

\begin{description}
  \item[Step 1.] Initially, let $D(k,u)$ indicate the minimum objective value of the restricted problem of separation \\
  \vspace{10pt}
  problem, where $k\in \{1,2,\ldots,v\}$ and $u\in \{0,1,\ldots,v\}$.
  \item[Step 2.] Given the initial conditions $D(1,0) = P$ and $D(1,1) = t_1 - \beta_1 +P$. The boundary conditions are \\
  \vspace{10pt}
  $D(k,u) = \infinity$ if $u > k$, for all $k \in V$.
  \item[Step 3.] Given the recursion:
  \begin{equation*}
  D(k,u)= \min \left\{
  \begin{aligned}
  & D(k-1,u), \text{for the case when} \ s^* \ \text{does not contain} \ k, \\
  & D(k-1,u-1) + u t_k - \alpha_k ,\text{for the case when} \ s^* \ \text{contains} \ k.
  \end{aligned}
  \right.
  \end{equation*}

  \item[Step 4.] Obtain the optimal objective value of separation problem by
  $\delta_{IVPU} = \min\{D(v,u): u\in \{1,2,\ldots,v-1\}\}$.
   return $\delta_{IVPU}$.
\end{description}

\end{algorithmic}
\end{algorithm}
