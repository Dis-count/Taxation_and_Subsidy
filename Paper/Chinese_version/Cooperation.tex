\documentclass[UTF8]{article}
\author {Discount}

\title {非平衡博弈的一个实例-机器调度博弈}
\date{}
\usepackage{ctex}
\usepackage{amsmath}

\usepackage{geometry}
\geometry{a4paper,scale=0.8}
\usepackage{graphicx}
\usepackage{amssymb}

\usepackage{setspace}
\renewcommand{\baselinestretch}{1.5}


\usepackage{float}
\usepackage{color}%,soul}f
\usepackage{multirow}
\usepackage{xr}


\begin{document}
    \maketitle

\begin{abstract}
    我们知道合作博弈
    我们使用了经典的列生成算法对一类经典的机器调度博弈进行了计算

list: 合作博弈的非空问题的解决方法现状;
      机器调度博弈的特殊性;
      列生成算法应用


可以再加入的内容:加入不同的相似 game

                加入算法解决,例如动态规划  贪婪算法  启发式算法

                对比 shapley 值以及 由算法求解得到的 分配解  进行比较 

\end{abstract}

\qquad \textbf{关键词: 合作博弈、费用分摊、列生成算法、机器调度博弈}

\section{引言}

我们知道合作博弈




$$
f(x)=\frac{1}{\sqrt{2 \pi \sigma x}} e^{-\frac{(x-\mu)^{2}}{2 \sigma^{2}}}
$$




\end{document}
